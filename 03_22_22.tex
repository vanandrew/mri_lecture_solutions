\documentclass{article}
\usepackage{amsmath}
\usepackage{amssymb}
\usepackage{gensymb}
\usepackage{mathtools}
\usepackage[margin=1in]{geometry}

\usepackage{titlesec}

\titleclass{\section}{top}
\newcommand\sectionbreak{\clearpage}

\begin{document}

\subsection*{1}

The right corner of voxel 3 is located at $x = 45$ mm
and $y = 45$ mm, while the origin ($x = 0$ mm, $y = 0$ mm) is
centered at the middle of voxel 5. Assuming the voxels are all
the same size, we can see easily see that the distance to the boundary
of the image is given by $s_{boundary} = 1.5s_{voxel}$, where $s_{boundary} = 45$
mm and $s_{voxel}$ is the length of a side of a voxel. Then with
some simple arithmetic, we can see that $s_{voxel} = 30$ mm. From that fact,
we can easily generate the coordinates for each voxel:

\begin{align*}
    &\textrm{Voxel 1: } x = -30 \textrm{ mm, } y = 30 \textrm{ mm}\\
    &\textrm{Voxel 2: } x = 0 \textrm{ mm, } y = 30 \textrm{ mm}\\
    &\textrm{Voxel 3: } x = 30 \textrm{ mm, } y = 30 \textrm{ mm}\\
    &\textrm{Voxel 4: } x = -30 \textrm{ mm, } y = 0 \textrm{ mm}\\
    &\textrm{Voxel 5: } x = 0 \textrm{ mm, } y = 0 \textrm{ mm} \\
    &\textrm{Voxel 6: } x = 30 \textrm{ mm, } y = 0 \textrm{ mm}\\
    &\textrm{Voxel 7: } x = -30 \textrm{ mm, } y = -30 \textrm{ mm}\\
    &\textrm{Voxel 8: } x = 0 \textrm{ mm, } y = -30 \textrm{ mm}\\
    &\textrm{Voxel 9: } x = 30 \textrm{ mm, } y = -30 \textrm{ mm}
\end{align*}

\subsection*{2}
At $t = 0.3$ ms, an RF pulse has been applied, but no gradients have been applied yet. Since all voxels are on
resonance, they experience the same phase. A $90\degree$ RF pulse in the $+x'$ will tip the magnetization vectors into
the $+y'$ direction. This means that each magnetization vector has a phase of $90\degree$.

\subsection*{3}
At $t = 0.6$ ms, a gradient in $x$ direction at $29.359$ mT/m was already applied for $0.1$ ms. Using
the equation: \[ \tilde{\phi} = 2\pi \bar{\gamma}G_{x}xt \]
We can solve for the phase of the magnetization at each voxel. Note that since
voxels (1, 4, 7), (2, 5, 8), and (3, 6, 9) respectively have the
same $x$ position, they will have the same phase. Note that since the magnetization vector rotations are clockwise, we
need to add a negative to the resulting angle to get the phase in the correct direction.  
\subsubsection*{Voxels 1, 4, 7}
\begin{align*}
    \tilde{\phi} &= 
    2\pi \left(42.6 * 10^{6} \frac{Hz}{T}\right)
    \left(29.359 * 10^{-3} \frac{T}{m} \right)
    \left(-30 * 10^{-3} m \right)
    \left(0.1 * 10^{-3} s \right) \\
    &= -23.5750 \textrm{ radians} = -1350.75\degree \approx 89\degree \xrightarrow{\text{clockwise}} -89\degree \\
    &\textrm{Adding the existing 90\degree phase, and the total phase at these voxels is 1\degree.}
\end{align*}
\subsubsection*{Voxels 2, 5, 8}
\begin{align*}
    \tilde{\phi} &= 
    2\pi \left(42.6 * 10^{6} \frac{Hz}{T}\right)
    \left(29.359 * 10^{-3} \frac{T}{m} \right)
    \left(0 * 10^{-3} m \right)
    \left(0.1 * 10^{-3} s \right) \\
    &= 0 \textrm{ radians} = 0\degree \\
    &\textrm{Adding the existing 90\degree phase, and the total phase at these voxels is 90\degree.}
\end{align*}
\subsubsection*{Voxels 3, 6, 9}
\begin{align*}
    \tilde{\phi} &= 
    2\pi \left(42.6 * 10^{6} \frac{Hz}{T}\right)
    \left(29.359 * 10^{-3} \frac{T}{m} \right)
    \left(30 * 10^{-3} m \right)
    \left(0.1 * 10^{-3} s \right) \\
    &= 23.5750 \textrm{ radians} = 1350.75\degree \approx 271\degree \xrightarrow{\text{clockwise}} -271\degree \\
    &\textrm{Adding the existing 90\degree phase, and the total phase at these voxels is -181\degree or 179\degree.}
\end{align*}

\subsection*{4}
At $t = 1.2$ ms, an additional gradient in the $y$ direction at $19.572$ mT/m was already applied for $0.3$ ms. Using
the equation: \[ \tilde{\phi} = 2\pi \bar{\gamma}G_{y}yt \]
We can solve for the phase of the magnetization at each voxel. Note that since
voxels (1, 2, 3), (4, 5, 6), and (7, 8, 9) respectively have the
same $y$ position, they will have the same additional phase added. Then we need to add the
phase from the $y$ gradient to the phase from the $x$ gradient to get the total
phase of the magnetization at each voxel. Again, since the magnetization vector rotations are clockwise, we
need to add a negative to the resulting angle to get the phase in the correct direction.  
\subsubsection*{Voxels 1, 2, 3}
\begin{align*}
    \tilde{\phi} &= 
    2\pi \left(42.6 * 10^{6} \frac{Hz}{T}\right)
    \left(19.572 * 10^{-3} \frac{T}{m} \right)
    \left(30 * 10^{-3} m \right)
    \left(0.3 * 10^{-3} s \right) \\
    &= 47.15 \textrm{ radians} = 2701.41\degree \approx 181\degree \xrightarrow{\text{clockwise}} -181\degree \textrm{ or } 179\degree
\end{align*}
\subsubsection*{Voxels 4, 5, 6}
\begin{align*}
    \tilde{\phi} &= 
    2\pi \left(42.6 * 10^{6} \frac{Hz}{T}\right)
    \left(19.572 * 10^{-3} \frac{T}{m} \right)
    \left(0 * 10^{-3} m \right)
    \left(0.1 * 10^{-3} s \right) \\
    &= 0 \textrm{ radians} = 0\degree
\end{align*}
\subsubsection*{Voxels 7, 8, 9}
\begin{align*}
    \tilde{\phi} &= 
    2\pi \left(42.6 * 10^{6} \frac{Hz}{T}\right)
    \left(19.572 * 10^{-3} \frac{T}{m} \right)
    \left(-30 * 10^{-3} m \right)
    \left(0.3 * 10^{-3} s \right) \\
    &= -47.15 \textrm{ radians} = -2701.41\degree \approx 179\degree \xrightarrow{\text{clockwise}} -179\degree \textrm{ or } 181\degree
\end{align*}
\subsection*{}
Now we add the phases from each gradient and the RF pulse to get the total phase at each voxel at $t = 1.2$ ms:
\begin{align*}
    &\textrm{Voxel 1: } 1\degree + 179\degree = 180\degree \\
    &\textrm{Voxel 2: } 90\degree + 179\degree = 269\degree \\
    &\textrm{Voxel 3: } 179\degree + 179\degree = 358\degree \\
    &\textrm{Voxel 4: } 1\degree + 0\degree = 1\degree \\
    &\textrm{Voxel 5: } 90\degree + 0\degree = 90\degree \\
    &\textrm{Voxel 6: } 179\degree + 0\degree = 179\degree \\
    &\textrm{Voxel 7: } 1\degree + 181\degree = 182\degree \\
    &\textrm{Voxel 8: } 90\degree + 181\degree = 271\degree \\
    &\textrm{Voxel 9: } 179\degree + 181\degree = 360\degree \textrm{ or } 0\degree
\end{align*}
Their phases are all unique!

\end{document}