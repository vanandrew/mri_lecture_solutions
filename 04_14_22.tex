\documentclass{article}
\usepackage{amsmath}
\usepackage{amssymb}
\usepackage{gensymb}
\usepackage{mathtools}
\usepackage[margin=1in]{geometry}

\usepackage{titlesec}

% \titleclass{\section}{top}
% \newcommand\sectionbreak{\clearpage}

\begin{document}

\section{While keeping the bandwidth of the RF pulse the same, which of the following would you do to excite a thinner slice? Please explain.}
The bandwidth of the slice is given by the lower and upper frequencies of the RF pulse. should
\begin{align*}
    \Delta f = f_{upper} - f_{lower}
\end{align*}

Each frequency is determined by the larmor precession equation $f = \bar{\gamma}B$, then:
\begin{align*}
    \Delta f = \bar{\gamma}B_{upper} - \bar{\gamma}B_{lower} = \bar{\gamma}\left(B_{upper} - B_{lower}\right)
\end{align*}

And the upper and lower $B$ field strengths are determined by $B_{z} = B_0 + G_z z$:
\begin{align*}
    \Delta f &= \bar{\gamma}\left(B_0 + G_z z_{upper} - B_0 - G_z z_{lower}\right) \\
    &= \bar{\gamma}\left(G_z z_{upper} - G_z z_{lower}\right) \\
    &= \bar{\gamma}G_z\left(z_{upper} - z_{lower}\right)
\end{align*}

From this expression, we can see that if we kept $\Delta f$ the same, but wanted a thinner slice
(which implies $z_{upper} - z_{lower}$ decreases), we must increase, $G_z$, the strength of the slice
gradient.

\section{In a multi-slice acquisition, which one(s) of the following are changing?}
\begin{itemize}
    \item RF bandwidth
    \item Gradient amplitude
    \item Center frequency of the RF pulse
\end{itemize}
The Gradient Amplitude and RF Bandwidth should remain the same, since we aren't changing the thickness of the
slice. However, the center frequency of the RF pulse would need to change for each slice acquired in the multi-slice
acquisition.

\section{If I would like to image the exact same slice at 1.5T and 3T, which one(s) of the following will be changing?}
\begin{itemize}
    \item RF bandwidth
    \item Gradient amplitude
    \item Center frequency of the RF pulse
\end{itemize}
From the 1st question we can see that the slice thickness is determined solely by the
RF Bandwidth and the Gradient Amplitude, so changing the $B_0$ field shouldn't change these values.

However, the center of frequency of the RF pulse will need to change since it is dependent on the $B_0$ field
through the larmor precession equation: $f = \bar{\gamma}B$.

\section{Remembering the following convention: \\
x-axis = Right to left of patient \\
y-axis = Posterior to anterior of patient \\
z-axis = Head to foot of patient \\
along which axis would you run the gradient field to excite:}
\begin{itemize}
    \item A coronal slice: The y-gradient
    \item A sagittal slice: The x-gradient
    \item A transversal slice: The z-gradient
    \item An oblique slice that is parallel to the AC-PC line in the brain: Some combination of the y and z gradients
\end{itemize}

\section{While exciting a slice, which one do you run first?}
\begin{itemize}
    \item RF pulse
    \item Gradient field
    \item You run them together
\end{itemize}
You must run them together, the gradient field selects the slice to be excited, while the RF pulse
actually excites the slice.

\end{document}