\documentclass{article}
\usepackage{amsmath}
\usepackage{amssymb}
\usepackage{gensymb}
\usepackage{mathtools}
\usepackage{hyperref}
\usepackage[margin=1in]{geometry}
\hypersetup{
    colorlinks=true,
    linkcolor=blue,
    filecolor=magenta,      
    urlcolor=blue,
    pdftitle={Overleaf Example},
    pdfpagemode=FullScreen,
    }

\usepackage{titlesec}

% \titleclass{\section}{top}
% \newcommand\sectionbreak{\clearpage}

\begin{document}
\section*{Derivation of EPI Distortion}
This derivation is based off of the \href{https://pubmed.ncbi.nlm.nih.gov/7674900/}{Jezzard and Balaban 1995} paper.

We first start with part of Equation 2 from the paper, the below equation represents the phase term for the exponential
term of an EPI experiment:
\begin{align*}
    2\pi\gamma\left(\int_{0}^{t}G_r(t)x dt + \int_{0}^{t}G_{pe}(t)y dt + \int_{0}^{t}\Delta B_0(x,y,z) dt\right)
\end{align*}

The paper then defines $\Delta \phi_r$ and $\Delta \phi_{pe}$ to be the distance between each k-space point. For the
readout, this implies the time difference between two adjacent points in the readout. For the phase-encode, this will
be the time between the same x-points between the current and next readout lines.

First, we consider $\Delta \phi_r$. Between adjacent points of the readout, the time is given by the dwell time. So for
some time in the readout $t^{*}$, we have its adjacent point $t^{*} + t_{dwell}$, where $t_{dwell}$ is the dwell time. We note
that $G_{pe}$ is 0 during this time, so that entire term goes to 0. This gives us:
\begin{align*}
    \Delta \phi_r &= 2\pi\gamma\left(\int_{0}^{t^{*} + t_{dwell}}G_r(t)x dt + \int_{0}^{t^{*} + t_{dwell}}\Delta B_0(x,y,z) dt\right) -
    2\pi\gamma\left(\int_{0}^{t^{*}}G_r(t)x dt + \int_{0}^{t^{*}}\Delta B_0(x,y,z) dt\right) \\
    &= 2\pi\gamma\left((t^{*} + t_{dwell})G_r(t)x + (t^{*} + t_{dwell})\Delta B_0(x,y,z)\right) -
    2\pi\gamma\left(t^{*}G_r(t)x + t^{*}\Delta B_0(x,y,z)\right) \\
    &= 2\pi\gamma\left(t_{dwell}G_r(t)x + t_{dwell}\Delta B_0(x,y,z)\right)
\end{align*}

Now we consider $\Delta \phi_{pe}$. The time between readout lines is given by the dwell time multiplied by twice the number of
points in the readout axis of the FOV (which the paper defines as $N$), this is the time from the start of the current readout line
to the end time of the next readout line. Adjacent points in $k_y$ are structured this way, due to the "zig-zag"-like trajectory
of an EPI experiment. So the time difference between readout lines is $2Nt_{dwell}$. However, the paper also considers
the ramp time of the phase-encoding gradients, so this time is augmented by $\tau_{ramp}$. So the total time between points
is $2Nt_{dwell} + 2\tau_{ramp}$. We also note that since the $k_x$ doesn't change between these two samples, it is
easy to see that the readout Gradient will sum to 0 during this time, so we can ignore the $G_r$ term. So for some time,
$t^{*}$ corresponding to the beginning of a readout line, and the time corresponding to the end of the next readout line,
$t^{*} + 2Nt_{dwell} + 2\tau_{ramp}$, we have:
\begin{align*}
    \Delta \phi_{pe} &= 2\pi\gamma\left(\int_{0}^{t^{*} + 2Nt_{dwell} + 2\tau_{ramp}}G_{pe}(t)y dt + \int_{0}^{t^{*} + 2Nt_{dwell} + 2\tau_{ramp}}\Delta B_0(x,y,z) dt\right) - \\
    &2\pi\gamma\left(\int_{0}^{t^{*}}G_{pe}(t)y dt + \int_{0}^{t^{*}}\Delta B_0(x,y,z) dt\right) \\
    &G_{pe} \textrm{ is only active for } \tau_{ramp} \textrm { for each readout line.} \\
    &= 2\pi\gamma\left(2\tau_{ramp}G_{pe}(t)y + (t^{*} + 2Nt_{dwell} + 2\tau_{ramp})\Delta B_0(x,y,z)\right) - \\
    &2\pi\gamma\left(\tau_{ramp}G_{pe}(t)y dt + t^{*}\Delta B_0(x,y,z) dt\right) \\
    &= 2\pi\gamma\left(\tau_{ramp}G_{pe}(t)y + (2Nt_{dwell} + 2\tau_{ramp})\Delta B_0(x,y,z)\right)
\end{align*}
\end{document}
